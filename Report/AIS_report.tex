\documentclass[10pt,a4paper]{article}
\usepackage[utf8]{inputenc}
\usepackage[english]{babel}
\usepackage{amsmath}
\usepackage{amsfonts}
\usepackage{amssymb}
\usepackage{graphicx}

\title{Title}
\author{Carlo Alessi}

\usepackage{soul}
\newcommand{\todo}[1]{\hl{\textbf{TODO:} #1}}

\begin{document}
\maketitle

\section{Introduction}
\todo{give some background on what is an earthquake.}

The Italian penisula has a long history of destructive earthquakes that dates back to at least the 62 AD, when a strong earthquake caused major damage to the city of Pompeii. It is believed that, after the strong earthquake of the 62 AD, a series of minor seisms eventually resulted in the eruption of the Mount Vesuvius in the 79 AD, which completely destroyed the city\footnote{https://en.wikipedia.org/wiki/62\_Pompeii\_earthquake}. 
Since then there have been an uncountable number of earthquakes, especially in the central part of the penisula, which is the focus of the study performed in this paper.

The outline of the paper is as follows. In section \ref{sec:central_italy} I describe the problem of earthquakes occurring in the central part of Italy, considering the implications of the geographical position as well as the economical and social effects they cause. \todo{ maybe, social, political, technical, environmental issues.} In section \ref{sec:features} are described the essential features needed to analyze the problem. Section \ref{sec:method} discusses about the possibility of applying different Artificial Intelligence techniques to solve the problem \todo{maybe, and breafly describes the chosen method to be applied on a real dataset}. In section \ref{sec:results} are reported the results of a small numeber of experiments. Section \ref{sec:results} concludes with a discussion and an insight for future work.

\section{Environmental Problem: Analysis of Earthquakes in central Italy} \label{sec:central_italy}
In this section I describe the problem of earthquakes in Umbria, a region of central Italy. \todo{describe why Umbria has earthquakes due to geographical reasons}.

With tha actual technology and expertise it is extremely challenging to predict the exact time, location and intensity of an earthquake. A possible approach could be prevention. 

Many monuments and buildings in the Appennines were constructed in ancient times and cannot resist severe shocks. Restructoring the buildings to make them anti-seismic would cost less than rebuilding them from scratch. However the actual political instability and the general economic crisis of Italy  overshadow this topic.



\begin{itemize}
\item specific environmental problem in your home city/region/country
\item get as much information about the problem as you can using reports from local/regional/national governmental agencies, NGOs or other associations or institutions, local newspapers, etc.
\item Describe the problem in detail and show its complexity.
\end{itemize}

tourists may be scared away.


\section{Characteristics of the problem}\label{sec:features}

\section{Method}\label{sec:method}

\subsection{Applicable techniques}

\section{Results}\label{sec:results}

\section{Conclusions}\label{sec:conclusions}


\end{document}